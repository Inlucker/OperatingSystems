\documentclass[a4paper,oneside,14pt]{extreport}

\usepackage[T2A]{fontenc}
\usepackage[utf8]{inputenc}
\usepackage[english,russian]{babel}

\usepackage[left=30mm, right=20mm, top=20mm, bottom=20mm]{geometry}

\usepackage{microtype}
\sloppy

\usepackage{setspace}
\onehalfspacing

\usepackage{indentfirst}
\setlength{\parindent}{12.5mm}

\usepackage{titlesec}
\titleformat{\chapter}{\LARGE\bfseries}{\thechapter}{20pt}{\LARGE\bfseries}
\titlespacing*{\chapter}{\parindent}{*2}{*2}
\titleformat{\section}{\Large\bfseries}{\thesection}{20pt}{\Large\bfseries}

\addto{\captionsrussian}{\renewcommand*{\contentsname}{Содержание}}
\usepackage{natbib}
\renewcommand{\bibsection}{\chapter*{Список использованных источников}}

\usepackage{caption}

\usepackage{wrapfig}
\usepackage{float}

\usepackage{graphicx}
\newcommand{\imgwc}[4]
{
	\begin{figure}[#1]
		\center{\includegraphics[width=#2]{inc/img/#3}}
		\caption{#4}
		\label{img:#3}
	\end{figure}
}
\newcommand{\imghc}[4]
{
	\begin{figure}[#1]
		\center{\includegraphics[height=#2]{inc/img/#3}}
		\caption{#4}
		\label{img:#3}
	\end{figure}
}
\newcommand{\imgsc}[4]
{
	\begin{figure}[#1]
		\center{\includegraphics[scale=#2]{inc/img/#3}}
		\caption{#4}
		\label{img:#3}
	\end{figure}
}

\usepackage{pgfplots}
\pgfplotsset{compat=newest}

\usepackage{listings}
\usepackage{listingsutf8}
\lstset{
	basicstyle=\footnotesize\ttfamily,
	keywordstyle=\color{blue},
	stringstyle=\color{red},
	commentstyle=\color{gray},
	numbers=left,
	numberstyle=\tiny,
	numbersep=5pt,
	frame=false,
	breaklines=true,
	breakatwhitespace=true,
	inputencoding=utf8/koi8-r
}

\lstdefinestyle{c}{
	language=C++,
	backgroundcolor=\color{white},
	basicstyle=\footnotesize\ttfamily,
	keywordstyle=\color{blue},
	stringstyle=\color{red},
	commentstyle=\color{gray},
	directivestyle=\color{orange},
	numbers=left,
	numberstyle=\tiny,
	stepnumber=1,
	numbersep=5pt,
	frame=single,
	tabsize=4,
	captionpos=t,
	breaklines=true,
	breakatwhitespace=true,
	escapeinside={\#*}{*)},
	morecomment=[l][\color{magenta}]{\#},
	columns=fullflexible
}

\newcommand{\code}[1]{\texttt{#1}}

\usepackage{amsmath}
\usepackage{amssymb}

\usepackage[unicode]{hyperref}
\hypersetup{hidelinks}

\makeatletter
\newcommand{\vhrulefill}[1]
{
	\leavevmode\leaders\hrule\@height#1\hfill \kern\z@
}
\makeatother

\begin{document}

\include{title}

\section*{Функции обработчика прерывания от системного таймера для Windows}
\noindent
\textbf{По тику:}
\begin{itemize}
	\item инкрементирует счётчик системного времени;
	\item декрементирует остаток кванта текущего потока;
	\item декрементирует счетчик отложенных задач;
	\item ставит в очередь DPC объект диспетчера настройки баланса
	(этот диспетчер активизируется каждую секунду для возможной инициации событий,
	связанных с планированием и управлением памятью).
\end{itemize}
\textbf{По главному тику:}
\begin{itemize}
	\item Возвращает задействованный в системе объект ''событие'', который ожидает диспетчер настройки баланса.
\end{itemize}
\textbf{По кванту:}
\begin{itemize}
	\item инициализирует диспетчеризацию потоков путем постановки соответствующего объекта в очередь DPC.
\end{itemize}

\section*{Функции обработчика прерывания от системного таймера для Unix}
\noindent
\textbf{По тику:}
\begin{itemize}
	\item инкремент счетчика тиков аппаратного таймера;
	\item инкремент счетчика использования процессора текущим процессом: инкремент поля \texttt{p\_cpu} дескриптора текущего процесса на единицу, до максимального значения, равного 127;
	\item инкремент часов и других таймеров системы;
	\item декремент счетчика времени до отправления на выполнение отложенных вызовов;
	\item декремент кванта текущего потока.
\end{itemize}
\textbf{По главному тику:}
\begin{itemize}
	\item инициализация отложенных вызовов (см. пояснения ниже) функций, относящиеся к работе планировщика;
	\item инициализация отложенного вызова (см. пояснения ниже) процедуры wakeup, которая
	перемещает дескрипторы процессов из очереди «спящих» в
	очередь «готовых к выполнению»;
	
	Отложенные вызовы считаются обычными процедурами ядра и не должны выполняться с приоритетами прерываний. Поэтому обработчик прерываний таймера не выполняет эти вызовы на прямую. На каждом тике обработчик прерываний таймера проверяет, не нужно ли начать выполнение отложенного вызова. Если он находит ожидающий вызоа, то выставляет флаг, указывающий на необходимость запуска \textit{обработчика отложенного вызова}.
	
	\item декремент счетчика времени, оставшегося до отправления одного из сигналов: 
	\begin{itemize}
		\item SIGALARM - сигнал, посылаемый процессу по истечении действительного времени;
		\item SIGPROF - сигнал, посылаемый процессу по истечении времени заданном в таймере профилирования;
		\item SIGVTALARM - сигнал, посылаемый процессу по истечении времени, заданного в «виртуальном» таймере.
	\end{itemize}
\end{itemize}
\textbf{По кванту:}
\begin{itemize}
	\item посылка текущему процессу сигнала SIGXCPU, если он израсходовал выделенный ему квант процессорного времени.
\end{itemize}

\end{document}